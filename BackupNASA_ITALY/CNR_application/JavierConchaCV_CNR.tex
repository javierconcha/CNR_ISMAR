\documentclass[11pt]{res}
%\usepackage{helvetica} % uses helvetica postscript font (download helvetica.sty)
%\usepackage{newcent}   % uses new century schoolbook postscript font 
\setlength{\textheight}{9.5in} % increase text height to fit resume on 1 page
\newsectionwidth{0pt}  % So the text is not indented under section headings
\usepackage{multicol}
% \usepackage{times}

\usepackage{hyperref}
\hypersetup{
    colorlinks=true,
    linkcolor=blue,
    filecolor=magenta,      
    urlcolor=cyan,
}
 
\urlstyle{same}

\usepackage[UKenglish]{datetime}

\begin{document} 
{\centering \Large \bf JAVIER A. CONCHA, Ph.D.\par} % the \\[12pt] adds a blank line after name
\vspace{0.3in}                                         
\begin{multicols}{2}
{\raggedright Postdoctoral Research Fellow\\ Institute of Marine Science (ISMAR)\\ National Research Council of Italy (CNR)\\ ~~~~~~~~~~~~~~~~~~~~~~~~~~~~~~~~~~~\\}
{\raggedleft Via Acqui 10, int. 8A\\Roma, RM 00183, Italy\\Mobile: +39 347 752 2354\\Email: Javier.Concha@artov.ismar.cnr.it\\}
\end{multicols}
\vspace{-0.4in} 
\hrulefill
\begin{resume}
%%%%%%%%%%%%%%%%%%%%%%%%%%%%%%%%%%%%%%%%%%%%%%%%%%%%%%%%%%%%%  
\vspace{-0.3in}                                         
\section{EDUCATION}
\vspace{0.1in}
{\bf Rochester Institute of Technology (RIT)}, Rochester, NY, USA\\
Ph.D. Candidate, Imaging Science Department, GPA 3.6, December {\bf 2015}\\
Thesis title: ``The Use of Landsat-8 for Monitoring of Fresh and Coastal Water,'' sponsored by the United States Geological Survey (USGS) contract G12PC00065, Advisor: Dr. John Schott\\
\vspace{0.1in}\\
{\bf Rochester Institute of Technology}, Rochester, NY, USA.\\
M.S., Imaging Science Department, GPA 3.5, August {\bf 2012}\\
Project Title: ``Atmospheric Compensation for WV2 Satellite and In-Water Component Retrieval,'' Advisor: Dr. John Schott
\vspace{0.1in}\\
{\bf University of Concepcion}, Concepcion, Chile \\
B.S., {\it with honors}, Electronics Engineering, October {\bf 2008}\\
%%%%%%%%%%%%%%%%%%%%%%%%%%%%%%%%%%%%%%%%%%%%%%%%%%%%%%%%%%%%%
\vspace{-0.1in}
\section{WORK EXPERIENCE}
\vspace{0.1in}
{\bf Istituto di Scienze Marine, Consiglio Nazionale delle Ricerche}, Rome, Italy\\
{\bf Postdoctoral Researcher Fellow}, position 005-2018-AR-RM, {\bf 2019--present}\\
Supervisor: Dr. Vittorio E. Brando\\
{\bf Main activities and responsibilities}: Developing and improving Ocean Color algorithms and Satellite-derived validation data using in situ data. Participant of the COSIMO 2020 field campaign. Installation of the PANTHYR and HYPSTAR\textsuperscript{\tiny\textregistered}/HYPERNETS systems at the Acqua Alta Oceanographic Tower (AAOT).\\
Specifics:
\begin{itemize}\setlength\itemsep{0em}
\item CMEMS OC-TAC: Acted as Technical Leader for the CMEMS OC-TAC (\url{https://marine.copernicus.eu/about/producers/oc-tac}).
\item H2020 HYPERNETS Project: Establishing the satellite-derived data validation using the HYPSTAR\textsuperscript{\tiny\textregistered} instrument.
\end{itemize}

% \vspace{0.1in}\\
{\bf NASA Goddard Space Flight Center/USRA}, Greenbelt, MD\\
{\bf Postdoctoral Researcher Fellow}, Ocean Ecology Lab, {\bf 2016--2018}\\
{\bf Main activities and responsibilities:} At NASA GSFC I worked on GOCI geostationary ocean color satelite data . In  particular, I was involved in the Vicarious Calibration of GOCI for the inclusion in SeaDAS  for the Ocean Color Retrieval (Concha et al., 2018) and on the analysis of Diurnal Variability of Biogeochemical Processes in the China Sea from geostationary  Remote Sensing Reflectance (Concha et al. 2019),\\
Description: Atmospheric correction and vicarious calibration for the Geostationary Ocean Color Imager (GOCI), development and validation of ocean color products for GOCI using SeaDAS/l2gen. Calculation of uncertainties in assessing diurnal variability using GOCI. Two fields campaigns.\\
Supervisor: Dr. Antonio Mannino\\
Supervisor's Address: NASA Goddard Space Flight Center, 8800 Greenbelt Rd, Greenbelt, MD 20771, USA\\
Supervisor's Phone: +1 (301) 286-0182\\
\vspace{0.1in}\\
{\bf Rochester Institute of Technology}, Rochester, NY\\
{\bf Instructor}, ``Intro to Instrumentation and Field Measurements in Remote Sensing'' 4-day course, graduate level, Imaging Science Department, Intersession Term, {\bf 2015}\\
{\bf Graduate Research Assistant}, Imaging Science Department, {\bf 2011--2015}\\
{\bf Graduate Teaching Assistant}, ``Radiometry'' and  ``Imaging Science Fundamentals'' courses, undergraduate level, Imaging Science Department, September {\bf 2010}--May {\bf 2011}
\vspace{0.1in}\\
{\bf University of Concepcion}, Concepcion, Chile\\
{\bf Research Assistant}, Geophysics Department, University of Concepcion, 2009--2010\\
{\bf Scientific Crew Member}, Center for Quantum Optics and Quantum Information, 2006--2008\\
{\bf Collaborator Member}, Itinerant Exhibition: ``The Universe of Light,'' Explora CONICYT Program, Center for Quantum Optics and Quantum Information, September--November {\bf 2007}\\
%%%%%%%%%%%%%%%%%%%%%%%%%%%%%%%%%%%%%%%%%%%%%%%%%%%%%%%%%%%%%
\vspace{-0.2in}
\section{RESEARCH INTERESTS}
Since my Ph.D., I’ve been exploring the use of several observation classes (e.g. MODIS/Aqua, Sentinel-3A/MSI, Landsat-8/OLI, COMS/GOCI) for studying processes at different spatial and temporal scales over optically complex waters. I have experience working with high spatial resolution sensors (i.e. Landsat-8/OLI) over inland waters from my Ph.D. thesis and medium spatial resolution sensors over coastal water from my Postdoc (i.e. COMS/GOCI). Currently, I am working on the inclusion of Sentinel 3 A and B Full Resolution (FR) data streams in the CMEMS catalogue. I am also involved in the validation protocols for the H2020 HYPERNETS project, a new generation of hyperspectral instruments for satellite data validation and calibration. I am currently studying quantitatively the influence of the different validation approaches on the reported validation statistics.
Scientific Research Interests:
I am interested in the calibration of satellite sensors and in developing, improving and validating Ocean Colour algorithms over optically complex water bodies. I use satellite sensors with diverse spatial, temporal and spectral resolution to study processes occurring a different scales. For instance, I have used high spatial resolution sensors (i.e. Landsat-8/OLI and Sentinel-2/MSI) over inland waters, and geostationary sensors (i.e. COMS/GOCI) to study diurnal biogeochemical processes in coastal waters.


%%%%%%%%%%%%%%%%%%%%%%%%%%%%%%%%%%%%%%%%%%%%%%%%%%%%%%%%%%%%%
% \vspace{-0.2in}
\section{SCHOLARSHIPS AND AWARDS}
\vspace{0.1in}
{\bf Fulbright Scholarship}, Rochester Institute of Technology, {\bf 2010--2012}
\vspace{0.1in}\\
{\bf {\bf 2014} Student of the Year Award, Graduate Prize}, Central New York Region, American Society for Photogrammetry and Remote Sensing (ASPRS), {\bf 2014}
\vspace{0.1in}\\
{\bf NASA and University of Maine Grant}, Grant to attend the summer course ``Calibration and Validation for Ocean Color Remote Sensing", Darling Marine Center, University of Maine, July {\bf 2013}
\vspace{0.1in}\\
{\bf Officer Travel Grant Award}, SPIE Optics+Photonics, August 8-13, San Diego, CA, USA, {\bf 2015}
\vspace{0.001in}\\
{\bf Travel Grant Award}, International Ocean Color Science meeting, June 15-18, San Francisco, CA, {\bf 2015}
\vspace{0.1in}\\
{\bf Best Poster Award}, III Annual Meeting of Chilean Scientists in the U.S., October 5-6, Columbia University, New York, NY, {\bf 2012}
\vspace{0.1in}\\
{\bf Travel Grant Award}, IEEE International Geoscience And Remote Sensing Symposium (IGARSS), 22-27 July, Munich, Germany, {\bf 2012}
\vspace{0.1in}\\
{\bf Merit Scholarship}, for Tuition, Imaging Science Department, Rochester Institute of Technology, {\bf 2010}
\vspace{0.1in}\\
{\bf RIT Graduate Department Scholarship}, for Tuition, Imaging Science Department, Rochester Institute of Technology, {\bf 2010}
\vspace{0.1in}\\
{\bf FONDECYT Scholarship}, Scholarship holder, Project FONDECYT 1061046: Experiments on Quantum Information Protocols with Two-Photon States from Spontaneous Parametric Down Conversion, Center for Quantum Optics and Quantum Information, University of Concepcion, Concepcion, Chile, {\bf 2006}\\

%%%%%%%%%%%%%%%%%%%%%%%%%%%%%%%%%%%%%%%%%%%%%%%%%%%%%%%%%%%%%
\vspace{-0.1in}
\section{VOLUNTEERING AND LEADERSHIP}
\vspace{0.1in}
{\bf Review Panelist}, 2021 CLEO Conference, A\&T 3 Optical Instrumentation for Measurements and Monitoring sub-committee, January, {\bf 2021}
\vspace{0.1in}\\
{\bf Review Panelist}, SPIE Scholarship Committee, {\bf 2018--2020}
\vspace{0.1in}\\
{\bf Review Panelist}, NASA ROSES-2017 A.37 The Science of Terra, Aqua, and Suomi NPP Peer Review OCEAN, October 16-18, Greenbelt, MD, USA, {\bf 2017}
\vspace{0.1in}\\
{\bf Review Panelist}, Carbon Cycle Focus Area, NASA Earth Science \& Space Fellowship (NESSF) program, April 20, Washington, DC, USA, {\bf 2017}
\vspace{0.1in}\\
{\bf Facilitator}, Student Chapter Leadership Workshop, SPIE Optics+Photonics, August, San Diego, CA, USA, {\bf 2015--2016}
\vspace{0.1in}\\
{\bf Attendee}, Student Chapter Leadership Workshop, SPIE Optics+Photonics, August 8-9, San Diego, CA, USA, {\bf 2015}
\vspace{0.1in}\\
{\bf Student Volunteer}, IEEE International Geoscience And Remote Sensing Symposium (IGARSS), July 13-18, Quebec, Canada, {\bf 2014}
\vspace{0.1in}\\
{\bf Secretary}, SPIE Student Chapter, Rochester Institute of Technology, {\bf 2014--2015}
\vspace{0.1in}\\
{\bf Executive Board Member and Co-founder}, Latin Rhythm Dance Club, Rochester Institute of Technology, {\bf 2011--2012}
\vspace{0.1in}\\
{\bf Treasurer}, Student Council, Electrical Engineering Department, University of Concepcion, Chile, {\bf 2005}\\
%%%%%%%%%%%%%%%%%%%%%%%%%%%%%%%%%%%%%%%%%%%%%%%%%%%%%%%%%%%%%
%%%%%%%%%%%%%%%%%%%%%%%%%%%%%%%%%%%%%%%%%%%%%%%%%%%%%%%%%%%%%
\vspace{-0.1in}
\section{PUBLICATIONS}
{\bf Summary}: Author of six journal papers, five conference papers and two technical reports.\\
{\bf Citation metrics -- Google Scholar}: Career citations of 162, h-index of 6, i10-index of 2 (\url{https://tinyurl.com/JC-CV-scholar}, 30/04/2021)\\
\vspace{-0.3in}   
\section{PUBLICATIONS -- PEER REVIEWED}
\vspace{0.05in}
{\bf Concha, J.A.}, Bracaglia, M., Brando, V.E., {\it Assessing the influence of different validation protocols on Ocean Colour match-up analyses}, Remote Sensing of Environment, {\bf 2021}.  Vol. 259. (\url{https://doi.org/10.1016/j.rse.2021.112415}).

Giardino, C., Bresciani, M., Braga, F., Fabbretto, A., Ghirardi, N., Pepe, M., Gianinetto, M., Colombo, R., Cogliati, S., Ghebrehiwot, S., Laanen, M., Peters, S., Schroeder, T., {\bf Concha, J.A.}, Brando, V.E., {\it First Evaluation of PRISMA Level 1 Data for Water Applications}, Sensors. {\bf 2020}. Vol. 20(16):4553. (\url{https://doi.org/10.3390/s20164553}).

{\bf Concha, J.A.}, Mannino, A., Franz, B. $\&$ Kim, W., {\it Uncertainties in the Geostationary Ocean Color Imager (GOCI) Remote Sensing Reflectance for Assessing Diurnal Variability of Biogeochemical Processes}, Remote Sensing, {\bf 2019}. Vol. 11(3):295 (\url{https://doi.org/10.3390/rs11030295}).

{\bf Concha, J.A.}, Mannino, A., Franz, B., Bailey, S., $\&$ Kim, W., {\it Vicarious Calibration of GOCI for SeaDAS Ocean Color Retrieval}, International Journal of Remote Sensing, {\bf 2018}. Vol. 40, issue 10: p. 3984-4001. (\url{https://doi.org/10.1080/01431161.2018.1557793}).

{\bf Concha, J.A.} $\&$ Schott, J., {\it Retrieval of color producing agents in Case 2 waters using Landsat 8}, Remote Sensing of Environment, {\bf 2016}. Vol. 185: p. 95-107. (\url{https://doi.org/10.1016/j.rse.2016.03.018})

Muller‐Karger, F.E., Hestir, E., Ade, C., Turpie, K., Roberts, D.A., Siegel, D., Miller, R.J., Humm, D., Izenberg, N., Keller, M., Morgan, F., Frouin, R., Dekker, A.G., Gardner, R., Goodman, J., Schaeffer, B., Franz, B.A., Pahlevan, N., Mannino, A.G., {\bf Concha, J.A.}, Ackleson, S.G., Cavanaugh, K.C., Romanou, A., Tzortziou, M., Boss, E.S., Pavlick, R., Freeman, A., Rousseaux, C.S., Dunne, J., Long, M.C., Klein, E., McKinley, G.A., Goes, J., Letelier, R., Kavanaugh, M., Roffer, M., Bracher, A., Arrigo, K.R., Dierssen, H., Zhang, X., Davis, F.W., Best, B., Guralnick, R., Moisan, J., Sosik, H.M., Kudela, R., Mouw, C.B., Barnard, A.H., Palacios, S., Roesler, C., Drakou, E.G., Appeltans, W.and Jetz, W., {\it Satellite sensor requirements for monitoring essential biodiversity variables of coastal ecosystems}, Ecological Applications, {\bf 2018}. Vol. 28, issue 3: p. 3984-4001. (\url{https://doi.org/10.1002/eap.1682})

%%%%%%%%%%%%%%%%%%%%%%%%%%%%%%%%%%%%%%%%%%%%%%%%%%%%%%%%%%%%%
% \vspace{-0.2in}
\section{PUBLICATIONS -- PROCEEDINGS}
\vspace{0.05in}
{\bf Concha, J.A.} $\&$ Schott, J., {\it Atmospheric Correction for Landsat 8 over Case 2 Waters}, Proc. SPIE 9607, Earth Observing Systems XX, {\bf 2015}. Vol. 9607. (\url{https://doi.org/10.1117/12.2188345})
\vspace{0.1in}\\
{\bf Concha, J.A.} $\&$ Schott, J., {\it In-water Component Retrieval over Case 2 Water using Landsat 8: Initial Results}, 2014 IEEE Geoscience and Remote Sensing Symposium, {\bf 2014}. pp. 4458-4461 (\url{https://doi.org/10.1109/IGARSS.2014.6947481})
\vspace{0.1in}\\
{\bf Concha, J.A.} $\&$ Schott, J., {\it A Model-based ELM for Atmospheric Correction over Case 2 water with Landsat 8}, Proc. SPIE 9111, Ocean Sensing and Monitoring VI, {\bf 2014}. Vol. 911112 (\url{https://doi.org/10.1117/12.2050589})
\vspace{0.1in}\\
{\bf Concha, J.A.} $\&$ Gerace, A., {\it Atmospheric Compensation for WV2 Satellite and In-Water Component Retrieval}, 2012 IEEE International Geoscience and Remote Sensing Symposium, {\bf 2012}. pp. 2833-2836. (\url{https://doi.org/10.1109/IGARSS.2012.6350842})
\vspace{0.1in}\\
{\bf Concha, J.A.} $\&$ Gerace, A., {\it Atmospheric Compensation for WV2 Satellite and In-Water Component Retrieval}, Proc. SPIE 8390, Algorithms and Technologies for Multispectral, Hyperspectral, and Ultraspectral Imagery XVIII, {\bf 2012}. Vol 8390. (\url{https://doi.org/10.1117/12.918962})
%****************************
\section{TECHNICAL REPORTS}
\vspace{0.1in}
HYPERNETS Validation Protocol and Plan for Water and Land Products, Deliverable D7.1/D8.1, Version 1.0, August 30, {\bf 2019}
\vspace{0.1in}\\
HYPERNETS Multi-Mission Validation of Water Products V1, Deliverable D7.2, Version 1.0, March 25, {\bf 2021}
\vspace{0.1in}\\
OC-TAC CMEMS QUID
%****************************
\section{CONFERENCE PRESENTATIONS -- NO PROCEEDINGS}
\vspace{0.1in}
{\bf Concha, J.A.}, Brando, V., Bracaglia, M., {\it Assessing differences on uncertainty estimates introduced by match-up protocols}, 6th Sentinel-3 Validation Team Meeting, December 15-17, {\bf 2020}
\vspace{0.1in}\\
{\bf Concha, J.A.}, Mannino, A., Franz, B., Bailey, S. $\&$ Kim, W., {\it Vicarious Calibration of GOCI for NASA’s Ocean Color Algorithms}, AGU Fall Meeting, December 10-14, Washington, DC, USA, {\bf 2018}
\vspace{0.1in}\\
{\bf Concha, J.A.}, Mannino, A., Franz, B., Bailey, S. $\&$ Kim, W., {\it Diurnal Variability in Marine Biogeochemistry with the Geostationary Ocean Color Imager using SeaDAS/l2gen}, Ocean Optics XIV, The Oceanography Society (TOS), October 7-12, Dubrovnik, Croatia, {\bf 2018}
\vspace{0.1in}\\
{\bf Concha, J.A.}, Mannino, A., Franz, B., Bailey, S. $\&$ Kim, W., {\it Uncertainties in assessing Diurnal Variability with the Geostationary Ocean Color Imager (GOCI)}, Ocean Sciences, February 11-16, Portland, OR, USA, {\bf 2018}
\vspace{0.1in}\\
{\bf Concha, J.A.}, Mannino, A., Franz, B. $\&$ Kim, W., {\it Assessing Diurnal Variability of Biogeochemical Processes using the Geostationary Ocean Color Imager (GOCI)}, International Ocean Color Science meeting, May 15-18, Lisbon, Portugal, {\bf 2017}
\vspace{0.1in}\\
Mannino, A., Novak, M., Salisbury, J., Kim, W., {\bf Concha, J.A.}, Tzortziou, M. $\&$ Mulholland, M., {\it Diurnal variability in optical properties and carbon stocks as indicators of biogeochemical cycling}, International Ocean Color Science meeting, May 15-18, Lisbon, Portugal, {\bf 2017}
\vspace{0.1in}\\
Novak, M., Mannino, A. $\&$ {\bf Concha, J.A.}, {\it Discrete Biogeochemical Measurements Collected from the KORUS-OC Field campaign: An Assessment of Satellite Products using In Situ Data}, International Ocean Color Science meeting, May 15-18, Lisbon, Portugal, {\bf 2017}
\vspace{0.1in}\\
{\bf Concha, J.A.}, {\it NASA: Space and Earth Observation}, Earth and Space Sciences symposium, $7^{th}$ NEXOS Chile-USA Annual Meeting, November 18, Philadelphia, PA, USA, {\bf 2016}
\vspace{0.1in}\\
{\bf Concha, J.A.}, Mannino, A $\&$ Franz, B., {\it GOCI processing with SeaDAS: Validation of the Atmospheric Correction}, Ocean Optics XXIII, October 23-28, Victoria, BC, Canada, USA, {\bf 2016}
\vspace{0.1in}\\
{\bf Concha, J.A.} $\&$ Mannino, A, {\it Landsat 8’s Atmospheric Correction in SeaDAS: Comparison with AERONET-OC}, SPIE Optics and Photonics,  August 28-September 01, San Diego, CA, USA, {\bf 2016}
\vspace{-0.1in}\\
{\bf Concha, J.A.} $\&$ Schott, J., {\it Retrieval of Color Producing Agents in Case 2 Waters using Landsat 8}, International Ocean Color Science meeting, June 15-18, San Francisco, CA, USA, {\bf 2015}
\vspace{0.1in}\\
{\bf Concha, J.A.}, {\it Water Constituent Retrieval over Case 2 Water using Landsat 8: Initial Results}, 57th Annual Conference, International Association for Great Lakes Research (IAGLR), May 26-30, McMaster University, Hamilton, ON, Canada, {\bf 2014}
\vspace{0.1in}\\
{\bf Concha, J.A.}, {\it The Use of Landsat 8 for Monitoring of Fresh and Coastal Water}, Graduate Research and Creativity Symposium, April 18, Rochester Institute of Technology, {\bf 2014}
\vspace{0.1in}\\
{\bf Concha, J.A.}, {\it El Uso del Nuevo Satélite Landsat-8 para el Monitoreo de Agua Dulce y Costera}, January 1, Departamento de Geofísica, Universidad de Concepcion, Concepcion, Chile, {\bf 2014}
\vspace{0.1in}\\
{\bf Concha, J.A.}, {\it Atmospheric Compensation for WV2 Satellite and In-Water Component Retrieval}, III Annual Meeting of Chilean Scientist in the U.S., October 5-6, Columbia University, New York, NY, {\bf 2012}
\vspace{0.1in}\\
{\bf Concha, J.A.} $\&$ Gerace, A., {\it Atmospheric Compensation for WV2 Satellite and In-Water Component Retrieval}, Master Research Conference, SUNY Brockport University, Brockport, NY, USA, April 14, {\bf 2012}
\vspace{0.1in}\\
{\bf Concha, J.A.} $\&$ Gerace, A., {\it Atmospheric Compensation for WV2 Satellite and In-Water Component Retrieval}, $22^{nd}$ Annual Student and Faculty Conference, Great Lakes Research Consortium, State University of New York (SUNY) at Oswego, Oswego, NY, USA, March 30 -- 31, {\bf 2012}\\

%%%%%%%%%%%%%%%%%%%%%%%%%%%%%%%%%%%%%%%%%%%%%%%%%%%%%%%%%%%%%
\vspace{-0.2in}
\section{MEETINGS AND WORKSHOPS}
\vspace{0.1in}
H2020 HYPERNETS Project Plenary Meeting , on-line, January 19-21, {\bf 2021}
\vspace{0.1in}\\
H2020 HYPERNETS Project Review Meeting, National Physical Laboratory, Teddington, U.K., January 14-16, {\bf 2020}
\vspace{0.1in}\\
Future of Above Water Radiometry (FAR-1) Workshop, October 21-23, Brussels, Belgium, {\bf 2019}
\vspace{0.1in}\\
ESA Living Planet Symposium 2019, May 13-17, Milan, Italy, {\bf 2019}
\vspace{0.1in}\\
SPIE Optics and Photonics,  August 06-10, San Diego, CA, USA, {\bf 2017}
\vspace{0.1in}\\
2016 NASA Ocean Color Research Team (OCRT) Meeting,  May 02-04, Silver Spring, MD, USA, {\bf 2016}
\vspace{0.1in}\\
Ocean Optics XXII Conference, The Oceanography Society (TOS),  October 26-31, Portland, ME, USA, {\bf 2014}
\vspace{0.1in}\\
2014 NASA Ocean Color Research Team (OCRT) Meeting,  May 05-07, Silver Spring, MD, USA, {\bf 2016}
\vspace{0.1in}\\
The Great Lakes Workshop Series on Remote Sensing of Water Quality, First Workshop Meeting, March 12-13, NASA Glenn Research Center, Cleveland, OH, USA, {\bf 2014}
\vspace{0.1in}\\
The Great Lakes Workshop Series on Remote Sensing of Water Quality, Second Workshop Meeting, May 7-8, NOAA GLERL, Ann Arbor, MI, USA, {\bf 2014}\\
%%%%%%%%%%%%%%%%%%%%%%%%%%%%%%%%%%%%%%%%%%%%%%%%%%%%%%%%%%%%%
\vspace{-0.1in}
\section{SPECIAL TRAINING AND COURSES}
\vspace{0.1in}
{\it ESA Advanced Ocean Synergy Training Course}, sponsored by ESA and Technical University of Crete, November 4-8, Chania, Greece, {\bf 2019}
\vspace{0.1in}\\
{\it Calibration and Validation for Ocean Color Remote Sensing}, Summer Course, sponsored by NASA and University of Maine, Darling Marine Center, University of Maine, July, Maine, USA, {\bf 2013}
%%%%%%%%%%%%%%%%%%%%%%%%%%%%%%%%%%%%%%%%%%%%%%%%%%%%%%%%%%%%%
\vspace{-0.1in}
\section{INVITED PRESENTATIONS AND KEYNOTE}
\vspace{0.1in}
{\bf Concha, J.A.}, {\it Assessing the Influence of Different Validation Protocols on Ocean Colour Match-Up Analyses; And Deriving Uncertainties for Assessing the Diurnal Variability of Biogeochemical Processes using GOCI}, Remote Sensing Group Seminar, Plymouth Marine Laboratory (PML), July 23, U.K., on-line, {\bf 2020}
\vspace{0.1in}\\
{\bf Concha, J.A.}, {Assessing Differences on Uncertainty Estimates introduced by Match-Up Protocols}, CNR-ISMAR Seminar, February 13, Rome, Italy, {\bf 2020}
\vspace{0.1in}\\
{\bf Concha, J.A.}, {\it The Uncertainties of the Uncertainties: Assessing Difference in the Uncertainty Estimates introduced by Match-Up Protocols}, Future of Above Water Radiometry (FAR-1) Workshop, October 21-23, Brussels, Belgium, {\bf 2019}
\vspace{0.1in}\\
{\bf Concha, J.A.}, {\it  Vicarious Calibration of GOCI for NASA’s Ocean Color Algorithms}, CNR-ISMAR Seminar, March 19, {\bf 2019}
\vspace{0.1in}\\
{\bf Concha, J.A.}, {\it NASA: Universe, Earth and Oceans}, Ocean, Earth, and Atmospheric Sciences departmental seminar, Old Dominion University, April 5, Norfolk, VA, USA, {\bf 2018}
\vspace{0.1in}\\
{\bf Concha, J.A.}, {\it NASA: Universe, Earth and Oceans}, Centro de Estudios Avanzados en Zonas Áridas (CEAZA), Universidad Serena, April 9, La Serena, Chile, {\bf 2018}
\vspace{0.1in}\\
{\bf Concha, J.A.}, {\it NASA: Universe, Earth and Oceans}, Centro de Investigación en Recursos Naturales y Sustentabilidad (CIRENYS), Universidad Bernardo O'Higgins, April 10, Santiago, Chile, {\bf 2018}
\vspace{0.1in}\\
{\bf Concha, J.A.}, {\it NASA: Universe, Earth and Oceans}, Centro para el Estudio de Forzantes Multiples sobre Sistemas Socio-Ecologicos Marinos (musels), EULA, University of Concepcion, April 12, Concepcion, Chile, {\bf 2018}
\vspace{0.1in}\\
{\bf Concha, J.A.}, {\it NASA: Universe, Earth and Oceans}, IMO-DOCE, Oceanography Department, University of Concepcion, April 13, Concepcion, Chile, {\bf 2018}
\vspace{0.1in}\\
%%%%%%%%%%%%%%%%%%%%%%%%%%%%%%%%%%%%%%%%%%%%%%%%%%%%%%%%%%%%%
\vspace{-0.1in}
\section{PROFESSIONAL AFFILIATIONS}
\vspace{-0.1in}
\subsection{\it Current}
\vspace{-0.2in}
{\bf International Society for Optics and Photonics -- SPIE}
\vspace{0.1in}\\
{\bf The Oceanography Society -- TOS}
\vspace{-0.2in}
\subsection{\it Past}
\vspace{-0.2in}
{\bf American Society for Photogrammetry and Remote Sensing -- ASPRS}
\vspace{0.1in}\\
{\bf Geoscience and Remote Sensing Society, Institute of Electrical and Electronics Engineers -- GRSS-IEEE}
\vspace{0.1in}\\
{\bf International Association for Great Lakes Research -- IAGLR}
\vspace{0.1in}\\
{\bf National Society of Leadership and Success -- NSLS}
\vspace{0.1in}\\
{\bf RIT Fulbright Scholar Association -- RITFSA}: Advisory Board member
\vspace{0.1in}\\
{\bf Toastmasters International}

%%%%%%%%%%%%%%%%%%%%%%%%%%%%%%%%%%%%%%%%%%%%%%%%%%%%%%%%%%%%%
\section{TECHNICAL SKILLS}
\vspace{0.1in}
{\bf Field and Lab Measurements}: Radiometric Measurements, Color Producing Agents (Chlorophyll-{\it a}, Total Suspended Solids (TSS) and Colored Dissolved Organic Matters (CDOM)) Concentrations and Absorption Coefficients
\vspace{0.1in}\\
{\bf Programming}: Python, Unix shell scripting, IDL, Matlab
\vspace{0.1in}\\
{\bf Applications}: ENVI/IDL, Matlab, LaTeX, NASA's SeaDAS, ESA's SNAP, ACOLITE, Git Version Control
\vspace{0.1in}\\
{\bf Physics-based Models}: MODTRAN, HydroLight
\vspace{0.1in}\\
{\bf Operating Systems}: Unix/Linux, Mac OS X, Windows\\

%%%%%%%%%%%%%%%%%%%%%%%%%%%%%%%%%%%%%%%%%%%%%%%%%%%%%%%%%%%%%
\vspace{-0.1in}
\section{PROFESSIONAL INTERESTS}
\vspace{0.1in}
Remote Sensing, Water Quality, Ocean Color, Water Constituents Retrieval, Atmospheric Correction, Satellite Calibration, Landsat, Sentinel, High Spatial Resolution Satellites\\

%%%%%%%%%%%%%%%%%%%%%%%%%%%%%%%%%%%%%%%%%%%%%%%%%%%%%%%%%%%%%
\vspace{-0.1in}
\section{HOBBIES and INTERESTS}
\vspace{0.1in}
Latin Dances; Yoga; Public Speaking.


% \vspace{4cm}
% \begin{tabular}{@{}p{4.3in}p{2in}@{}}
% & \hrulefill \\
% \today, Greenbelt, MD, USA \hfill& ~~~Javier A. Concha, Ph.D.\\
% \end{tabular}
\end{resume}
\end{document}











